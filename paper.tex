\documentclass[runningheads,a4paper]{llncs}

\usepackage[american]{babel}

\usepackage{graphicx}

%extended enumerate, such as \begin{compactenum}
\usepackage{paralist}

%put figures inside a text
%\usepackage{picins}
%use
%\piccaptioninside
%\piccaption{...}
%\parpic[r]{\includegraphics ...}
%Text...

%Sorts the citations in the brackets
%\usepackage{cite}

%for easy quotations: \enquote{text}
\usepackage{csquotes}
\usepackage[T1]{fontenc}

%enable margin kerning
\usepackage{microtype}

%better font, similar to the default springer font
%cfr-lm is preferred over lmodern. Reasoning at http://tex.stackexchange.com/a/247543/9075
\usepackage[%
  rm={oldstyle=false,proportional=true},%
  sf={oldstyle=false,proportional=true},%
  tt={oldstyle=false,proportional=true,variable=true},%
  qt=false%
]{cfr-lm}
%

%if more space is needed, exchange cfr-lm by mathptmx
%\usepackage{mathptmx}

%for demonstration purposes only
\usepackage[math]{blindtext}

%enable hyperref without colors and without bookmarks
\usepackage[
  bookmarks=false,
  breaklinks=true,
  colorlinks=true,
  linkcolor=black,
  citecolor=black,
  urlcolor=black,
  pdfpagelayout=SinglePage
]{hyperref}

%enables correct jumping to figures when referencing
\usepackage[all]{hypcap}

%enable \cref{...} and \Cref{...} instead of \ref: Type of reference included in the link
\usepackage[capitalise,nameinlink]{cleveref}

%Nice formats for \cref
\crefname{section}{Sect.}{Sect.}
\Crefname{section}{Section}{Sections}
\crefname{figure}{Fig.}{Fig.}
\Crefname{figure}{Figure}{Figures}

\usepackage{xspace}
\newcommand{\eg}{e.\,g.,\ }
\newcommand{\ie}{i.\,e.,\ }

% correct bad hyphenation here
\hyphenation{op-tical net-works semi-conduc-tor}

\begin{document}

\title{Distributed Web Applications with IPFS, Tutorial}
\author{David Dias mail@daviddias.me \and Juan Benet juan@benet.ai}
\institute{Protocol Labs}

\maketitle

%% ABSTRACT

\begin{abstract}
IPFS, the InterPlanetary File System, is the distributed and permanent Web, a protocol to make the Web faster, more secure, open and available. IPFS could be seen as Git meets a BitTorrent swarm, exchanging objects within one Git repository. In other words, IPFS provides a high throughput content-addressed block storage model, with content-addressed hyperlinks. This forms a generalised MerkleDAG, a data structure that can beused to build versioned file systems, blockchains, unix like file systems, amongst other options. IPFS combines a Distributed Hash Table, an incentivised block exchange and a self-certifying namespace. IPFS has no single point of failure, and nodes do not need to trust each other.

This tutorial will focus on the IPFS Application Stack, including: libp2p, the networking layer; bitswap for data exchange;IPLD and the MerkleDAG, the thin waist data structure of IPFS and how to use IPFS interface to build distributed applications. The full length of the tutorial is 6 hours.
\end{abstract}

%% KEYWORDS

\keywords{IPFS, Web, Distributed, P2P, Cryptography, MerkleTree, MerkleDAG, IPLD, Go, JavaScript, Application, Apps, Blockchain, Hash, Secure, Data, File System, Files, Graphs, Database}

%% INTRO

\section{Introduction}\label{sec:intro}

\section{Motivations and goals}\label{sec:motivation}

\section{IPFS, the InterPlanetary FileSystem}\label{sec:ipfs}

%% TUTORIAL

\section{Tutorial}\label{sec:tutorial}

\subsection{Learning outcomes}

\subsection{Target audience}

\subsection{Curriculum}

%% PRESENTER

\section{Presenter}\label{sec:presenter}



%% CONCLUSION

\section{Conclusion}\label{sec:conclusion}

\subsection{Acknowledgments}


%% BIBLIOGRAPHY

\bibliographystyle{splncs03}
\bibliography{paper}

All links were last followed on March 10, 2016.

\end{document}




% ------------------------------------------------------------------------------

%% EXAMPLES

% Winery~\cite{Winery} is graphical modeling tool.

% \begin{figure}
% Simple Figure
% \caption{Simple Figure}
% \label{fig:simple}
% \end{figure}

% \begin{table}
% \caption{Simple Table}
% \label{tab:simple}
% Simple Table
% \end{table}

% cref Demonstration: Cref at beginning of sentence, cref in all other cases.

%\Cref{fig:simple} shows a simple fact, although \cref{fig:simple} could also show something else.
%\Cref{tab:simple} shows a simple fact, although \cref{tab:simple} could also show something else.
%\Cref{sec:intro} shows a simple fact, although \cref{sec:intro} could also show something else.

%Brackets work as designed:
%<test>

%\begin{inparaenum}
%\item All these items
%\item appear in one line
%\item This is enabled by the paralist package.
%\end{inparaenum}

%``something in quotes'' using plain tex or use \enquote{the enquote command}.

% \section{Conclusion and Outlook}
% \subsubsection*{Acknowledgments}
% E.g., \enquote{The 2\textsuperscript{nd} conference on examples}.



